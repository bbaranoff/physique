% Options for packages loaded elsewhere
\PassOptionsToPackage{unicode}{hyperref}
\PassOptionsToPackage{hyphens}{url}
%
\documentclass[
  11pt,
]{article}
\usepackage{amsmath,amssymb}
\usepackage{iftex}
\ifPDFTeX
  \usepackage[T1]{fontenc}
  \usepackage[utf8]{inputenc}
  \usepackage{textcomp} % provide euro and other symbols
\else % if luatex or xetex
  \usepackage{unicode-math} % this also loads fontspec
  \defaultfontfeatures{Scale=MatchLowercase}
  \defaultfontfeatures[\rmfamily]{Ligatures=TeX,Scale=1}
\fi
\usepackage{lmodern}
\ifPDFTeX\else
  % xetex/luatex font selection
\fi
% Use upquote if available, for straight quotes in verbatim environments
\IfFileExists{upquote.sty}{\usepackage{upquote}}{}
\IfFileExists{microtype.sty}{% use microtype if available
  \usepackage[]{microtype}
  \UseMicrotypeSet[protrusion]{basicmath} % disable protrusion for tt fonts
}{}
\makeatletter
\@ifundefined{KOMAClassName}{% if non-KOMA class
  \IfFileExists{parskip.sty}{%
    \usepackage{parskip}
  }{% else
    \setlength{\parindent}{0pt}
    \setlength{\parskip}{6pt plus 2pt minus 1pt}}
}{% if KOMA class
  \KOMAoptions{parskip=half}}
\makeatother
\usepackage{xcolor}
\usepackage[margin=2.5cm]{geometry}
\usepackage{graphicx}
\makeatletter
\def\maxwidth{\ifdim\Gin@nat@width>\linewidth\linewidth\else\Gin@nat@width\fi}
\def\maxheight{\ifdim\Gin@nat@height>\textheight\textheight\else\Gin@nat@height\fi}
\makeatother
% Scale images if necessary, so that they will not overflow the page
% margins by default, and it is still possible to overwrite the defaults
% using explicit options in \includegraphics[width, height, ...]{}
\setkeys{Gin}{width=\maxwidth,height=\maxheight,keepaspectratio}
% Set default figure placement to htbp
\makeatletter
\def\fps@figure{htbp}
\makeatother
\setlength{\emergencystretch}{3em} % prevent overfull lines
\providecommand{\tightlist}{%
  \setlength{\itemsep}{0pt}\setlength{\parskip}{0pt}}
\setcounter{secnumdepth}{5}
\usepackage{fontspec}
\setmainfont{Latin Modern Roman}
\usepackage{polyglossia}
\setmainlanguage{french}
\usepackage{csquotes}
\usepackage{amsmath,amssymb,amsthm}
\usepackage{booktabs,paralist,epigraph,hyperref}
\usepackage{tikz}
\usetikzlibrary{calc,shapes.geometric,arrows.meta,decorations.pathmorphing}
\ifLuaTeX
  \usepackage{selnolig}  % disable illegal ligatures
\fi
\IfFileExists{bookmark.sty}{\usepackage{bookmark}}{\usepackage{hyperref}}
\IfFileExists{xurl.sty}{\usepackage{xurl}}{} % add URL line breaks if available
\urlstyle{same}
\hypersetup{
  pdftitle={Principe d'unification spectrale : Mécanique Quantique et Relativité Générale},
  pdfauthor={Bastien Baranoff},
  hidelinks,
  pdfcreator={LaTeX via pandoc}}

\title{Principe d'unification spectrale : Mécanique Quantique et
Relativité Générale}
\usepackage{etoolbox}
\makeatletter
\providecommand{\subtitle}[1]{% add subtitle to \maketitle
  \apptocmd{\@title}{\par {\large #1 \par}}{}{}
}
\makeatother
\subtitle{Avec une analyse de la condition de Novikov}
\author{Bastien Baranoff}
\date{25 octobre 2025}

\begin{document}
\maketitle

\hypertarget{ruxe9sumuxe9}{%
\section{Résumé}\label{ruxe9sumuxe9}}

Le \textbf{principe d'unification spectrale} propose une approche
opératoire pour unifier la \textbf{mécanique quantique (MQ)} et la
\textbf{relativité générale (RG)}.\\
Cette théorie repose sur l'idée que \textbf{la matière et la géométrie}
ne sont que deux manifestations d'une \textbf{même densité spectrale}
\(S(\nu)\).\\
L'énergie, la masse et la courbure en découlent par simple intégration
fréquentielle.

\begin{center}\rule{0.5\linewidth}{0.5pt}\end{center}

\hypertarget{principe-de-cohuxe9rence-spectrale}{%
\section{1. Principe de cohérence
spectrale}\label{principe-de-cohuxe9rence-spectrale}}

Pour tout système quantique, on définit une \textbf{densité spectrale
normalisée} :

\[
\int S(\nu)\, d\nu = 1.
\]

L'énergie-masse effective du système découle de sa distribution
fréquentielle :

\[
rho = frac{1}{c^2} \int h\nu\, S(\nu)\, d\nu.
\]

Ainsi, la masse \(m\) n'est plus un invariant fondamental, mais une
\textbf{moyenne spectrale pondérée}.\\
Les états discrets de la mécanique quantique correspondent à des modes
dominants du spectre.

\begin{center}\rule{0.5\linewidth}{0.5pt}\end{center}

\hypertarget{couplage-uxe0-la-relativituxe9-guxe9nuxe9rale}{%
\section{2. Couplage à la Relativité
Générale}\label{couplage-uxe0-la-relativituxe9-guxe9nuxe9rale}}

Cette densité spectrale \(S(\nu)\) agit comme \textbf{source du tenseur
énergie-impulsion} :

\[
T_{\mu\nu} = \rho\, u_\mu u_\nu.
\]

Les équations d'Einstein prennent alors la forme :

\[
R_{\mu\nu} - \tfrac{1}{2} R g_{\mu\nu} = \kappa\, \rho\, u_\mu u_\nu
\]

où \(\kappa = 8\pi G/c^4\).\\
La géométrie de l'espace-temps devient ainsi \textbf{l'expression
macroscopique de la distribution spectrale de l'énergie}.

\begin{center}\rule{0.5\linewidth}{0.5pt}\end{center}

\hypertarget{interpruxe9tation-physique}{%
\section{3. Interprétation physique}\label{interpruxe9tation-physique}}

\begin{itemize}
\item
  Les \textbf{singularités gravitationnelles} deviennent des
  \textbf{pics spectraux} :\\
  la densité \(S(\nu)\) se concentre localement, sans divergence
  géométrique.
\item
  Les \textbf{particules massives} sont des \textbf{ondes cohérentes}
  stationnaires dans l'espace-temps.\\
  La courbure locale \(R_{\mu\nu}\) encode leur phase spectrale.
\item
  La \textbf{continuité relativiste} émerge naturellement du spectre :\\
  la somme des fréquences crée la métrique effective \(g_{\mu\nu}\).
\end{itemize}

\begin{center}\rule{0.5\linewidth}{0.5pt}\end{center}

\hypertarget{champ-complexe-du-chemin}{%
\section{4. Champ complexe du chemin}\label{champ-complexe-du-chemin}}

Le système global est décrit par un \textbf{champ complexe} :

\(\psi(x,t) = \psi_\mathrm{R}(x,t) + i\, \psi_\mathrm{I}(x,t)\)

où : - \(\psi_\mathrm{R}\) représente la \textbf{réalité induite}
(mesurable), - \(\psi_\mathrm{I}\) la \textbf{réalité conduite}
(géométrique, gravitationnelle).

La rétroaction entre les deux composantes relie la dynamique quantique à
la déformation gravitationnelle :\\
la \textbf{phase du champ} devient le pont entre courbure et
probabilité.

\begin{center}\rule{0.5\linewidth}{0.5pt}\end{center}

\hypertarget{consuxe9quences-principales}{%
\section{5. Conséquences
principales}\label{consuxe9quences-principales}}

\begin{itemize}
\tightlist
\item
  Passage naturel du \textbf{discret (MQ)} au \textbf{continu (RG)}.\\
\item
  Disparition des singularités au profit de pics spectraux réguliers.\\
\item
  Le champ complexe du chemin agit comme un \textbf{backpropagation
  physique} :\\
  les erreurs locales d'énergie rétro-agissent sur la cohérence
  globale.\\
\item
  Une description unifiée des particules, ondes et géométries comme
  \textbf{formes d'auto-cohérence spectrale}.
\end{itemize}

\begin{center}\rule{0.5\linewidth}{0.5pt}\end{center}

\hypertarget{perspectives-et-applications}{%
\section{6. Perspectives et
applications}\label{perspectives-et-applications}}

\begin{itemize}
\tightlist
\item
  \textbf{Gravité quantique} : formulation sans renormalisation
  explicite.\\
\item
  \textbf{Cosmologie} : la structure du spectre \(S(\nu)\) pourrait
  remplacer la métrique FLRW.\\
\item
  \textbf{Information} : chaque champ \(S(\nu)\) définit un espace
  d'états en cohérence, mesurable par interférométrie fréquentielle.
\end{itemize}

Ces perspectives pourraient aboutir à une \textbf{géométrie spectrale de
l'espace-temps}, où la métrique \(g_{\mu\nu}\) émerge du spectre des
interactions cohérentes.

\begin{center}\rule{0.5\linewidth}{0.5pt}\end{center}

\hypertarget{ruxe9fuxe9rences}{%
\section{7. Références}\label{ruxe9fuxe9rences}}

\begin{itemize}
\tightlist
\item
  Einstein, A. (1920). \emph{Relativity: The Special and the General
  Theory.}\\
\item
  Dirac, P.A.M. (1930). \emph{The Principles of Quantum Mechanics.}\\
\item
  Wheeler, J.A. (1957). \emph{Geometrodynamics and the Quantum.}
  Rev.~Mod. Phys., 29, 463.\\
\item
  Baranoff, B. (2025). \emph{Action unifiée de Kaluza--Dirac--Einstein.}
\end{itemize}

\begin{center}\rule{0.5\linewidth}{0.5pt}\end{center}

\hypertarget{annexe-relations-fondamentales}{%
\subsection{8. Annexe : Relations
fondamentales}\label{annexe-relations-fondamentales}}

\[
E = h\nu
\rho = \frac{1}{c^2} \int h\nu, S(\nu)\, d\nu.
T_{\mu\nu} = \rho\, u_\mu u_\nu
R_{\mu\nu} - \tfrac{1}{2} R g_{\mu\nu} &= \kappa\, T_{\mu\nu}
\]

\begin{center}\rule{0.5\linewidth}{0.5pt}\end{center}

\begin{quote}
\emph{« La masse n'est qu'une onde figée dans la trame du temps,\\
et la courbure, son écho dans l'espace des fréquences. »}\\
--- B. Baranoff (2025)
\end{quote}

\end{document}
